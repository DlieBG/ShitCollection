	\documentclass[12pt,a4paper]{article}
%\addtolength{\footskip}{-1cm}
\usepackage[a4paper, left=1.0cm, right=1.0cm, top=1.0cm, bottom=1.0cm]{geometry}
\usepackage[utf8]{inputenc}
%\usepackage[onehalfspacing]{setspace}
\usepackage{mathptmx}
\usepackage[T1]{fontenc}
\usepackage[ngerman]{babel}
\usepackage{graphicx} 
\usepackage{hyperref} 
\usepackage{pgfplots}
\usepackage{tikz}
\usetikzlibrary{positioning,shadows,arrows}
\usepackage{amssymb}
\date{}
\title{CASIO $fx-9860GII$ \glqq Cheats\grqq \vspace{-2ex}}
\author{Florian Tünte}
%

%\usepackage[onehalfspacing]{setspace}
\def\g#1{\texttt{#1}}
\usepackage{pgfplots}

\begin{document}
	\maketitle
	
	\pagenumbering{gobble}
	
	
	
		\section{Copy/Cut \& Paste}
	Copy/Cut\\
	\g{SHIFT} \g{8}\\\\
	Paste\\
	\g{SHIFT} \g{9}
	
	
	
	\section{GRAPH}
	\glqq Dickes\grqq \space \g{Y} mit Zahl kombiniert $\Rightarrow$ Graph Gleichung\\
	$ \Rrightarrow $ z.B. in der Ableitungsfunktion \g{Y1} benutzen\\\\
	Ableitung zeichnen\\
	\g{OPTN} \g{F2} \g{F1}  mit \g{x=X}\\
	$ \Rrightarrow $ Ausprobieren ob Ableitung richtig ist\\\\
	Integral zeichnen\\
	\g{OPTN} \g{F2} \g{F3}  von \g{0} bis \g{X}\\\\
	Graph Aussehen\\
	\g{F4}
	
	
	
	 \section{EQUA}
	 
	\subsection{SIML}
	3 Gleichungen mit 2 Unbekannten eingeben indem man eine Spalte mit 1 en auf\"ullt.\\
	Entsprechender Parameter $=0$ $\Rightarrow$ L\"osung und Probe in einem, ansonsten keine Lösung
	
	\subsection{POLY}
	Aufpassen: Hinten in der Gleichung $=0$
	
	\subsection{SOLV}
	 \g{RCL} $\Rightarrow$ Graph Gleichungen zum einfügen\\
	 \g{Lower \& Upper} sind die Grenzen $\Rightarrow$ darauf achten das die Lösung darin liegt\\
	 Ableitung und Integral auch verfügbar unter \g{OPTN} \g{F2}
	 
	 
	 
	 \section{DYNA}
	 Funktionenschar und Ortskurve \"uberpr\"ufen
	 
	 
	 
	 \section{TABLE}
	 Funktionswerte für x in bestimmtem Bereich in bestimmten Schritten
	 
	 
	 
	 \section{RUN-MAT}
	 \g{Wert$\rightarrow$[Buchstabe]}\\
	 Weißt dem Buchstaben den Wert zu (einfacheres mehrfaches Einsetzen)\\
	 \glqq Dickes\grqq \space \g{Y} (\g{SHIFT} \g{ALPHA} \g{-}) mit Zahl kombiniert $\Rightarrow$ Graph Gleichung\\
	 $ \Rrightarrow $ mit \g{zahl$\rightarrow$X} kombinieren\\
	 
	 \subsection{CONV}
	 \g{OPTN} \g{F6} \g{F1}\\
	 \g{Zahl[Einheit]$ \vartriangleright $[Einheit]}
	 
	 \subsection{ESYM}
	 \g{OPTN} \g{F6} \g{F6} \g{F1}\\
	 \g{Wert[Maßstab]} $ \Rightarrow $ eigentlicher Wert\\
	 \g{Wert$\div$1[Maßstab]} $ \Rightarrow $ Wert im Maßstab
	 
	 \subsection{SolvN}
	 \g{OPTN} \g{F4} \g{F5} $\Rightarrow$ Gleichung mit mehreren Lösungen
	 
	 \subsection{Vct}
	 Wenn es das unter \g{F3} \g{F6} ein Vektor Menu erscheint Update schon gemacht\\
	 \g{OPTN} \g{F2}\\
	 \g{DotP(Vct A,Vct B)} $\Rightarrow$ Skalarprodukt\\
	 \g{CrsP(Vct A,Vct B)} $\Rightarrow$ Kreuzprodukt\\
	 \g{Angel(Vct A,Vct B)} $\Rightarrow$ Winkel\\
	 \g{Norm(Vct A)} $\Rightarrow$ L\"ange
	 
\end{document}
