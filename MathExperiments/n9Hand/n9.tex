	\documentclass[12pt,a4paper]{article}
\addtolength{\footskip}{-1cm}
\usepackage[a4paper, left=1.0cm, right=1.0cm, top=1.0cm, bottom=1.0cm]{geometry}
\usepackage[utf8]{inputenc}
\usepackage[onehalfspacing]{setspace}
\usepackage{mathptmx}
\usepackage[T1]{fontenc}
\usepackage[ngerman]{babel}
\usepackage{graphicx} 
\usepackage{hyperref} 
\usepackage{pgfplots}
\usepackage{tikz}
\usetikzlibrary{positioning,shadows,arrows}

\title{$Graph \leftrightarrow Ableitungsgraph$}
%\author{Florian Tünte}
%

\usepackage[onehalfspacing]{setspace}
\def\code#1{\texttt{#1}}
\usepackage{pgfplots}

\begin{document}
	\pagenumbering{gobble}
	\section{Graph $\leftrightarrow$ Ableitungsgraph}
	\subsection{Ableitung $\rightarrow$ Graph}
	Bedeutung der Ableitung beachten (positiv $\rightarrow$ Graph steigend | negativ $\rightarrow$ Graph fallend)\\
	Nullstellen der Ableitung $\rightarrow$ Extrema der Funktion\\
	Extrema der Ableitung $\rightarrow$ Wendepunkten der Funktion\\
	H\"ohe des y-Achsenabschnitts Beachten wenn im Text was von Startwert oder \"ahnliches steht, ansonsten komplett egal (Bei Integration das +c)\\
	
	\begin{figure}[htb]
	\centering
	\begin{minipage}[t]{.45\linewidth}
		\centering
	
	%\caption{$$f'(x)=$$}
	\begin{tikzpicture}[line cap=round,line join=round,>=triangle 45]
	
	\begin{axis}[
	x=0.7cm,y=0.5cm,
	axis lines=middle,
	%ymajorgrids=true,
	%xmajorgrids=true,
	xmin=-5,
	xmax=5,
	ymin=-4,
	ymax=3,
	%x filter/.expression={abs(x)<0.3 ? nan : x},
	restrict y to domain=-10:10,
	]
	\draw[line width=4.pt] (-15.43,7.94) -- (-11.43,7.94);
	\pgfmathsetmacro{\constante}{1}
	\addplot [red,domain=-15:15,samples=400]
	{1/20*(x+3)*(x-1)*(x-4)};
	\end{axis}
	\end{tikzpicture}
\end{minipage}
\hfill
	\begin{minipage}[t]{.45\linewidth}
		\centering
	\begin{tikzpicture}[line cap=round,line join=round,>=triangle 45]
	
	\begin{axis}[
	x=0.7cm,y=0.5cm,
	axis lines=middle,
	%ymajorgrids=true,
	%xmajorgrids=true,
	xmin=-5,
	xmax=5,
	ymin=-4,
	ymax=3,
	%x filter/.expression={abs(x)<0.3 ? nan : x},
	restrict y to domain=-10:10,
	]
	\draw[line width=4.pt] (-15.43,7.94) -- (-11.43,7.94);
	\pgfmathsetmacro{\constante}{1}
	\addplot [red,domain=-15:15,samples=400]
	{((x*x*x*x)/80 - (x*x*x)/30 - (11*x*x)/40 + 3*x/5)};
	\end{axis}
	\end{tikzpicture}
\end{minipage}
%\caption{Bildtitel}
\end{figure}
	
	
	
	\subsection{Graph $\rightarrow$ Ableitung}
	Bedeutung der Ableitung beachten (steigend $\rightarrow$ positive Ableitung | fallend $\rightarrow$ negative Ableitung)\\
	Extrema der Funktion $\rightarrow$ Nullstellen der Ableitung\\
	Wendepunkten der Funktion $\rightarrow$ Extrema der Ableitung\\
	\begin{figure}[htb]
		\centering
		\begin{minipage}[t]{.45\linewidth}
			\centering
	\begin{tikzpicture}[line cap=round,line join=round,>=triangle 45]
	
	\begin{axis}[
	x=0.7cm,y=0.5cm,
	axis lines=middle,
	%ymajorgrids=true,
	%xmajorgrids=true,
	xmin=-5,
	xmax=5,
	ymin=-4,
	ymax=3,
	%x filter/.expression={abs(x)<0.3 ? nan : x},
	restrict y to domain=-10:10,
	]
	\draw[line width=4.pt] (-15.43,7.94) -- (-11.43,7.94);
	\pgfmathsetmacro{\constante}{1}
	\addplot [red,domain=-15:15,samples=400]
	{1/20*(x+3)*(x+2)*(x-1)*(x-4)};
	\end{axis}
	\end{tikzpicture}
\end{minipage}%
\hfill%
\begin{minipage}[t]{.45\linewidth}
	\centering
	\begin{tikzpicture}[line cap=round,line join=round,>=triangle 45]
	
	\begin{axis}[
	x=0.7cm,y=0.5cm,
	axis lines=middle,
	%ymajorgrids=true,
	%xmajorgrids=true,
	xmin=-5,
	xmax=5,
	ymin=-4,
	ymax=3,
	%x filter/.expression={abs(x)<0.3 ? nan : x},
	restrict y to domain=-10:10,
	]
	\draw[line width=4.pt] (-15.43,7.94) -- (-11.43,7.94);
	\pgfmathsetmacro{\constante}{1}
	\addplot [red,domain=-15:15,samples=400]
	{(x/20 + 3/20)*(x - 4)*(x - 1) + (x/20 + 3/20)*(x - 4)*(x + 2) + (x/20 + 3/20)*(x - 1)*(x + 2) + (x - 4)*(x - 1)*(x + 2)/20};
	\end{axis}
	\end{tikzpicture}
\end{minipage}
\end{figure}
	
\subsection{Zum \"Uben}
\begin{figure}[htb]
	\centering
	\begin{minipage}[t]{.45\linewidth}
		\centering
		\begin{tikzpicture}[line cap=round,line join=round,>=triangle 45]
		
		\begin{axis}[
		x=0.7cm,y=0.5cm,
		axis lines=middle,
		%ymajorgrids=true,
		%xmajorgrids=true,
		xmin=-5,
		xmax=5,
		ymin=-5,
		ymax=5,
		%x filter/.expression={abs(x)<0.3 ? nan : x},
		restrict y to domain=-10:10,
		]
		\draw[line width=4.pt] (-15.43,7.94) -- (-11.43,7.94);
		\pgfmathsetmacro{\constante}{1}
		\addplot [red,domain=-15:15,samples=400]
		{-1/2*(x+1)*(x-2)*(x+3)};
		\end{axis}
		\end{tikzpicture}
	\end{minipage}%
	\hfill%
	\begin{minipage}[t]{.45\linewidth}
		\centering
		\begin{tikzpicture}[line cap=round,line join=round,>=triangle 45]
		
		\begin{axis}[
		x=0.7cm,y=0.5cm,
		axis lines=middle,
		%ymajorgrids=true,
		%xmajorgrids=true,
		xmin=-5,
		xmax=5,
		ymin=-5,
		ymax=5,
		%x filter/.expression={abs(x)<0.3 ? nan : x},
		restrict y to domain=-10:10,
		]
		\draw[line width=4.pt] (-15.43,7.94) -- (-11.43,7.94);
		\pgfmathsetmacro{\constante}{1}
		%\addplot [red,domain=-15:15,samples=400]
		%{(x/20 + 3/20)*(x - 4)*(x - 1) + (x/20 + 3/20)*(x - 4)*(x + 2) + (x/20 + 3/20)*(x - 1)*(x + 2) + (x - 4)*(x - 1)*(x + 2)/20};
		\end{axis}
		\end{tikzpicture}
	\end{minipage}
\end{figure}
	
\end{document}
